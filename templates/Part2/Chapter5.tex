    \chapter{Exploitation (future) de la donnée}

En plus de réfléchir à la ou les bases de données à utiliser, il est également nécessaire de penser à l’utilisation future de la donnée. En effet, il convient de choisir les outils et les interfaces de façon à pouvoir répondre aux besoins des utilisateurs. Dans le cadre du projet sur la formation de l’Europe au XIIème siècle, la future plateforme a pour vocation d’être utilisée spécifiquement dans le cadre de la recherche. Le public visé se compose donc principalement de chercheur.se.s et étudiant.e.s. L’enjeu ici est de pouvoir répondre à des questions de recherche.



    \section{Durabilité des technologies}

Nous avons déjà évoqué la question de la durabilité des technologies lors de la présentation des différents types de bases de données. C’est un enjeu important dans le choix des technologies, notamment pour des projets s’étirant sur plusieurs années comme celui du schisme alexandrin.
En effet, certaines technologies peuvent être abandonnées à un temps donné. Pour illustrer ce propos, nous examinerons trois exemples de technologies devenues impopulaires voire obsolètes: un logiciel, un langage et un format de données.

    \subsection{Adobe Flash Player}
    
Flash Player était un logiciel permettant de visualiser du contenu multimédia ( vidéos, images) sur un navigateur web. Fin 2017, Adobe annonce que Flash ne sera plus maintenu, puis fin 2020, Flash est totalement déprécié\footnote{Terme utilisé en programmation, indique qu'un système n'est plus utilisé ou a été remplacé par quelque chose de plus moderne} et n’est plus supporté par l’ensemble des navigateurs web.
La plateforme Flash était utilisée dans différents domaines, comme par exemple dans des services d’archives ou encore en littérature électronique\footnote{Oeuvre littéraire nativement numérique et supposée être lu par un ordinateur. Heckman David, O’Sullivan James, \textit{Electronic Literature: Contexts and Poetics} in \textit{Literary Studies in the Digital Age}, 2018, \url{https://dlsanthology.mla.hcommons.org/electronic-literature-contexts-and-poetics/.}}. Dans ce dernier cas, Flash Player permettait aux auteur.rice.s d’expérimenter avec des outils de créations\footnote{Salter Anastasia, Murray John, \textit{E-Lit after Flash: The Rise (and Fall) of a “Universal” Language}, in Grigar, Dene et O’Sullivan James \textit{Electronic Litterature as Digital Humanities: contexts, forms and pratices}, BloomsBury Academic, 2021, p267-274}. 


    \subsection{Ruby}

Ruby est un langage de programmation orienté objet créé durant les années 1990. Il connut un succès conséquent pendant quelques années, notamment car le langage  est considéré plus flexible que ses concurrents et qu’il permet d’augmenter la productivité des développeurs. Ruby est essentiellement utilisé pour du développement web, notamment avec  son framework le plus populaire: Ruby on Rails (RoR). Ruby est utilisé dans plusieurs projets d’humanités numériques, comme le Deutsche Inschriften au CCeH par exemple.
Ruby est un langage qui perd de plus en plus d’intérêt auprès des développeurs, à cause de son écosystème complexe, de ses faibles performances et de la concurrence rude d'autres langages (Node.JS, Python). 
Comme on peut le constater dans les diagrammes en annexe(\ref{annexe:Ruby_stats}), Ruby est un langage assez peu utilisé, il apparaît même 16ème parmi les langages utilisés par les sondé.e.s en 2023. Par ailleurs, on constate que les utilisateur.rice.s préfèrent Node JS ou encore Python à Ruby. 


    
    \subsection{JPEG2000}

Le JPEG 2000 est un format présenté en 2000 qui permet notamment de compresser une image sans perdre en qualité, il est très utilisé en archives ou en bibliothèques. Par exemple, la Bibliothèque Nationale de France (BnF) a remplacé progressivement le format TIFF par le JPEG2000 depuis 2015\footnote{Cavalié Etienne, Caron Bertrand, \textit{Formats de données pour la préservation à long terme : la politique de la BnF}, version 3 du 9 septembre 2021, pages 81, p46.}. Ce dernier est maintenant utilisé comme format de référence pour la numérisation des plans à la BnF\footnote{\textit{Ibid}.}.\\
Aujourd’hui, plusieurs questions se posent: JPEG2000 n’est supporté que par très peu de navigateurs, a des soucis pour s’adapter à l'Open Source et est un format assez complexe\footnote{\textit{Lack of performant open source decoding software}, Open Preservation Foundation, https://wiki.opf-labs.org/display/TR/Lack+of+performant+open+source+decoding+software }. Beaucoup considèrent ce format comme une technologie de niche, difficile à implémenter. L’intégrer dans un nouveau projet pourrait s’avérer risqué.\\

Ces trois exemples nous permettent de saisir l’importance de l’évaluation de la durabilité des technologies considérées pour notre projet. Pour ce faire, les axes suivants sont à envisager:\\

\begin{itemize}
    \item Utiliser des logiciels Open Source, car il est plus probable qu’ils soient maintenues sur le long terme, voire repris par la communauté. De plus, cela permet de réduire le coût du projet.
    \item Estimer la pérennité d’une technologie en fonction de la taille de sa communauté. Par exemple, le XML TEI est une grande communauté, notamment avec les \textit{guidelines} et le TEI consortium, ce qui en fait, comme dit plus haut, un langage pérenne et un choix sûr. De la même manière, le SQL existe depuis près de 50 ans.\\
    
\end{itemize}


Le choix des technologies à utiliser dans un projet devient très vite un casse-tête, car il faut pouvoir à la fois choisir des logiciels et des langages accessibles aux chercheur.se.s qui n’ont pas spécialement de formation en informatique tout en répondant aux besoins très spécifiques du projet.


    \section{Faire le lien entre les chercheur.se.s et la donnée: UX/UI}

L’user experience (UX) est défini dans la norme ISO 9241-110:2010 comme les perceptions et réponses d’un individu découlant de l’utilisation et/ou la future utilisation d’un produit, d’un système ou d’un service\footnote{Allam, Hussin, Dahlan, \textit{User Experience: Challenges and Opportunities}.}. L’objectif est de prendre en compte les besoins des utilisateurs, ici principalement les chercheur.se.s tout en proposant une interface ergonomique. L’user interface (UI) correspond à la façon dont l’utilisateur visualise les données. \\
L’intérêt de s’intéresser à ces deux notions réside dans la volonté de faire adopter le projet et la plateforme par les utilisateurs. 
    
    \subsection{Interfaces graphiques}

L’interface graphique est un élément crucial du projet, sa qualité et sa facilité d’utilisation doit être une préoccupation majeure car une plateforme qui ne convient pas à l'utilisateur devient rapidement une plateforme inutilisée.\\ 

Dans le secteur privé, c’est l’UX/UI Designer qui se charge de la conception visuelle du produit. Il a pour mission de comprendre les besoins des utilisateurs et d’imaginer l’interface la plus claire pour répondre à ses besoins.
En humanités numériques, comme dans d’autres secteurs où les moyens financiers sont moins importants, cette responsabilité revient généralement au développeur. Cependant, celui-ci ne pourra être aussi compétent qu’une personne dont c’est le métier. Ainsi, lorsque c’est possible, allouer des fonds pour recourir à un expert du design paraît sensé.\\  
Le choix des technologies utilisées pour développer l’interface graphique n’aura pas d’impact sur l’interface elle-même, mais influera sur la simplicité de son développement et sur sa maintenabilité. Par ailleurs, sur un projet de 18 ans comme celui du CCeH, il est important d’envisager que la plateforme va évoluer, que ce soit à la demande des utilisateurs pour une amélioration de l’interface, ou pour l’ajout de fonctionnalités qui toucheront à l’interface. Les langages de base pour le développement d’interfaces web sont bien connus : \verb|HTML/CSS/JavaScript|, mais de nos jours, avec la complexification des interfaces à développer, c’est le choix des frameworks et librairies associés à ces langages qui importe le plus. Au CCeH, \verb|VueJS| semble être le framework \verb|JavaScript| le plus populaire. D’autres alternatives existent, comme par exemple \verb|React| ou \verb|Angular|. Chaque framework ou librairie apporte son lot d’avantages et inconvénients, le choix de l’un plutôt que l'autre dépend essentiellement de la familiarité des développeurs avec celui-ci. Néanmoins, certaines fonctionnalités peuvent faire pencher la balance. Par exemple,  laisse moins de latitude au développeur et offre une expérience de développement bien plus cadrée que \verb|VueJS| ou \verb|React|, ce qui peut constituer un avantage important dans un projet d’humanités numériques où les développeurs ne viennent que rarement du monde de l’informatique. Par ailleurs, en ce qui concerne les mises à jour, \verb|Angular| intègre des outils qui permettent de faciliter la montée de version en modifiant automatiquement le code, diminuant ainsi le coût de maintenance. \\ 
Il est également envisagé de donner un accès aux utilisateurs à la plateforme en s’inspirant du modèle du projet \href{https://www.itineranova.be/in/home}{Itinera Nova}. Initié en 2009 par les archives municipales du Louvain et développé par le CCeH, ce projet a pour objectif de numériser et mettre à disposition l’ensemble des registres des juges et les livres comptables de la ville de 1361 à 1795. Itinera Nova utilise actuellement un framework XRX, évoqué plus haut, et permet de gérer plusieurs niveaux d’accès à la plateforme. En effet, les transcriptions des documents d’archives se font par un groupe de volontaires, qui ont un compte utilisateur, puis sont vérifiées et validées par des administrateurs. Des réflexions autour de la mise en place d’un système de droits doivent donc être engagées.


    \subsection{Formats de données}


Ces dernières années, le JSON s’est imposé comme format de données principal pour le développement d’APIs. Il est simple à lire et à traiter informatiquement. Toutefois, il peine à trouver sa place dans le domaine des humanités numériques où le XML semble lui être préféré. Cela n’est pas un hasard, car même s’il est possible de représenter en JSON des graphes acycliques (arbres) comme le fait le XML, le résultat devient illisible pour un humain. Ainsi, certaines données sont mieux représentées en JSON et d’autres en XML. Par exemple, il conviendra d’utiliser le XML pour une édition numérique de charte, alors que le JSON serait préférable pour lister l’ensemble des lieux visités par un pape.\\
Pour autant, la plupart des APIs ne proposent qu’un format de données, mais ce n’est pas une fatalité: le protocole HTTP prévoit qu’un serveur puisse communiquer avec différents formats de données. L’en-tête HTTP Accept permet d’indiquer au serveur le format auquel on souhaite recevoir la donnée (par exemple : “Accept: application/json” ou “Accept: text/xml”). Plusieurs choix s’offrent alors à nous :\\
\begin{itemize} 
\item Se contenter d’un seul format de données, par manque de moyens ou de temps
\item Proposer l’ensemble de l’API développée aux formats JSON et XML, le client configurera sa préférence grâce à l’en-tête HTTP Accept
\item Segmenter les entités représentées par l’API selon qu’elles soient mieux représentées en JSON ou en XML et ne proposer qu’un seul format de retour par route d’API. On aurait ainsi les éditions numériques en XML et les données spatiales en JSON.
\end{itemize}

    
    \section{Favoriser la collaboration et l'échange}

 Le projet du schisme alexandrin devrait se matérialiser au travers d’une plateforme internationale et collaborative. Afin de garantir une interopérabilité, une connectivité et une utilisation durables des données,  nous appliquerons les principes FAIR: \textit{Findable, Accessible, Interoperable, and Re-usable}. Ces principes sont issus du mouvement de l’\textit{open science}, dont l’objectif est de permettre la réutilisabilité des données générées au cours des projets de recherche\footnote{Francesco Beretta, \textit{A challenge for historical research: Making data FAIR using a collaborative ontology management environment (OntoME)}, in \textit{Semantic Web}, 2021, p279-294.}.
Les principes FAIR sont utilisés afin de guider la production et la publication des données issues de la recherche\footnote{Beretta Francesco, \textit{Données ouvertes liées et recherche historique : un changement de paradigme}, 2023, \url{https://journals.openedition.org/revuehn/3349}}. \\
Une partie de ces principes ont finalement déjà été évoqué précédemment: \\
\begin{itemize}
\item \textit{Findable}: les données doivent avoir un identifiant unique et persistant, décrites avec des métadonnées riches, et indexées. Cette étape se retrouve dans la modélisation des données et dans le modèle de données.
\item \textit{Accessible}: les données doivent être récupérables par leur identifiant à l’aide d’un protocole standardisé et gratuit. Les métadonnées doivent être accessibles même si la donnée n’est plus disponible. Les APIs REST utilisent le protocole HTTP.\\
\end{itemize}

Dans le cadre de la collaboration et des échanges, les principes d'interopérabilité et de réutilisation des données sont primordiaux.



    \subsection{Interopérabilité}

L’interopérabilité doit faire partie des réflexions menées dans les choix technologiques pour les projets d’humanités numériques. Selon les principes FAIR, l’interopérabilité se caractérise de la façon suivante:\\
\begin{itemize} 
\item un langage qui permet la compréhension commune des objets numériques
\item des références à des données externes\\
\end{itemize}
Dans le cadre du projet du schisme alexandrin, il est prévu de permettre l’export des données au format RDF: un des langages préconisés par les principes FAIR. 
Il sera aussi possible d’accéder à la littérature du RI OPAC ainsi qu’à des liens vers le \href{https://viaf.org/}{VIAF}, le fichier d’autorité international virtuel.\\
De manière plus globale, la norme ISO/CEI 2382-18 définit l’interopérabilité comme étant l’aptitude de plusieurs unités fonctionnelles à coopérer pour traiter des données. Ainsi, choisir des protocoles de communication et des formats de données répandus et simples à implémenter permettra d’augmenter l’interopérabilité du système développé. Par exemple, le protocole SOAP - qui impose le XML comme format d’échange et est considéré complexe à implémenter - ne devait être le premier choix de quiconque souhaitant développer un système interopérable. Il faudra favoriser le REST, plus simple à implémenter et qui ne restreint pas l’usage d’un format de données spécifique.\\
La présence et même le format d’une documentation influe également sur l’interopérabilité du projet réalisé. En effet, impossible d’implémenter une API non documentée ou de tirer pleinement parti d’un outil non documenté. L’édition d’une documentation d’API à un format standardisé (OpenAPI) favorisera ainsi l’adoption du projet développé. Il est préférable, lorsqu’on implémente une API, de visualiser une documentation OpenAPI grâce à l’éditeur Swagger (editor.swagger.io) plutôt que de consulter une documentation au format PDF.

    \subsection{Réutiliser les données}

La réutilisation des données, c'est-à-dire leur utilisation à d’autres fins que celles pour lesquelles elles ont été initialement collectées, est particulièrement importante en science. En effet, cela permet à différents chercheur.se.s d’analyser et de publier indépendamment les uns des autres des résultats basés sur les mêmes données. La réutilisabilité est un élément clé des principes FAIR.\\
Afin de pouvoir rendre accessible les données et permettre leur réutilisation, les chercheur.se.s du projet du schisme alexandrin souhaitent s’appuyer sur le principe de \textit{Linked Open Data}. L’\textit{Open Data} ou la donnée ouverte est une donnée accessible, réutilisable et redistribuable sans condition \footnote{\url{https://www.opendatafrance.net/}}. Le \textit{Linked Open Data} désigne les relations entre les différentes ressources du Web, comme évoqué toute à l’heure avec la possibilité d’accéder à des ressources externes au projet. 