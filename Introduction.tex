\chapter{Introduction}

Les humanités numériques ont fait l’objet de plusieurs définitions. Elles peuvent être tout simplement définies comme la représentation de l’interception entre les technologies numériques et les sciences humaines et sociales \footnote{Drucker Johanna, Kim David, Salehian Iman, \textit{Introduction to Digital Humanities}, 2014, p119. \og \textit{Digital humanities is work at the intersection of digital technology and humanities disciplines} \fg{}}. Les humanités numériques sont une discipline transdisciplinaire, qui offrent de nouvelles perspectives grâce au développement d’outils numériques.\\
C'est précisément l'objectif du projet sur la formation de l'Europe au 12ème siècle à travers le schisme Alexandrin:  une collaboration entre le Cologne Center for eHumanities (CCeH), l'université d'Aachen et l'université de Würzburg. Les objectifs de ce projet sont vastes, à la fois en termes de durée et de méthodologie. Ils visent à examiner la période du schisme entre les années 1130 et 1181, une période jusqu'ici sous-étudiée du point de vue de son impact, au cours de laquelle la papauté a joué un rôle déterminant dans la formation de l'Europe. Les deux principaux objectifs sont les suivants:\\
\begin{itemize}
    \item Analyser de manière systématique le contenu des sources pour mieux comprendre le rôle central du schisme alexandrin dans la l'évolution  de l'Europe au XIIème siècle. L'étude se concentre notamment sur l'organisation des domaines de pouvoir de chaque protagoniste, les stratégies de communication, d'administration et juridiques.
    \item Mettre en valeur un vaste corpus de sources clairement défini, en utilisant des outils numériques et une plateforme de travail et de communication.\\
\end{itemize}

Avec une durée estimée à 18 ans et ayant commencé en avril de cette année, nous sommes à la genèse du projet du schisme alexandrin.Mon stage avait pour objectif de participer aux phases initiales du projet afin d'observer les défis spécifiques aux projets d’humanités numériques, en particulier ceux liés à la longévité des données. Pour répondre à ces questions, ce mémoire est divisé en trois parties. La première porte sur le principe de la modélisation des données dans le contexte des humanités numériques. Ensuite, nous examinerons les choix technologiques en jeu. Enfin, nous aborderons les problématiques liées à la pérennisation des données.


