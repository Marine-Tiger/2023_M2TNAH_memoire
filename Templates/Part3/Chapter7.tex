\chapter{L'archivage du projet}

Afin d’aborder cet thématique, nous allons prendre pour exemple le projet \textit{SustainLife}, qui  est un projet commun de l’Institut d’architecture des applications systèmes (IAAS) de l’université de Stuttgart, et du Data Center for the Humanities (DCH) de la Faculté des Arts et Humanités de l’université de Cologne. Son objectif est d’améliorer la pérennisation des projets d’humanités numériques\footnote{Neuefeind Claes, Schildkamp Philip, Mathiak Brigitte, Harzenetter Lukas, Barzen Johanna, Breitenbücher Uwe, Leymann Frank, \textit{The SustainLife Project – Living Systems in Digital Humanities}.}.

    \section{La norme \textit{OAIS}}
La norme ISO 14721 \textit{Open archival information system} (OAIS) correspond à un cadre conceptuel du cycle de vie des données et logiciels et identifie les tâches opérationnelles pour la préservation numérique. Elle est notamment employée dans le contexte de l’archivage électronique.\\ 
Le projet \textit{SustainLife} s’appuie notamment sur \textit{TOSCA}, un standard de la norme \textit{OAIS} pour la modélisation, l’approvisionnement et la gestion des applications \textit{cloud}\footnote{\textit{Ibid}.}. Pour implémenter ce standard, l’université de Stuttgart a implémenté \textit{OpenTOSCA}, qui est une sorte d’écosystème qui permet de sauvegarder dans le \textit{cloud} les projets conformes à un modèle de base compatible avec le standard TOSCA.

C'est une solution parmi plusieurs de sauvegarde de projets dans une application \textit{cloud}. L'avantage d'\textit{OpenTOSCA} est que c'est un écosystème \textit{Open Source}, et peut être imaginé spécifiquement pour les projets d'humanités numériques.
 