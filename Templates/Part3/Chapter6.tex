Nous avons déjà évoqué la question de la durabilité des technologies lors de la réflexion des choix technologiques et même lors de l’étape de modélisation de données. Nous pouvons constater à partir d’exemples cités plus haut qu'il est possible d’être confronté à la disparition des technologies de stockage, ce qui peut entraîner des coûts de maintenance supplémentaires\footnote{Huc Claude, \textit{La pérennisation des informations sous forme numérique : risques, enjeux et éléments de solution}, in Med Sci, 2008, p 653-658.}. Cette partie sera consacrée à penser à l’après-projet, c’est-à-dire la durabilité de la plateforme, puis la fin de ce projet et des projets en humanités numériques en général. C’est une question intéressante qui est la finalité de l’ensemble des problématiques de durabilité et de choix technologiques évoquées dans la partie précédente.
La pérennisation d’un projet est une notion plutôt vaste. Il faut commencer par se questionner sur ce que l’on doit conserver\footnote{Drucker Johanna, \textit{Sustainability and complexity: Knowledge and authority in the digital humanities}, in \textit{Digital Scholarship in the Humanities}, 2021, pp9.}. Archiver seulement les données d’un projet n’est pas suffisant, il faut également penser aux logiciels si l’on souhaite préserver l’ensemble du projet\footnote{Neuefeind Claes, Schildkamp Philip, Mathiak Brigitte, Harzenetter Lukas, Barzen Johanna, Breitenbücher Uwe, Leymann Frank, \textit{Technologienutzung im Kontext Digitaler Editionen – eine Landschaftsvermessung}, DHd 2019.}.

\chapter{Maintenance}

La fin du projet ne signifie pas forcément la fin de la plateforme. Afin de maintenir l’accès à celle-ci et aux différents outils, plusieurs possibilités s’offrent à nous: \\: 
\begin{itemize}
\item Permettre que le projet soit \textit{Open Source}. Accompagné d'une documentation technique, cela donne la possibilité que la plateforme soit maintenue par la communauté.
\item La redondance des données. Cela correspond à conserver à plusieurs endroits différents les mêmes données pour s'assurer que celles-ci soient toujours disponibles.
\item Sauvegarder le projet au \textit{Data Center for Humanities} de l'univeristé de Cologne.
\end{itemize}

