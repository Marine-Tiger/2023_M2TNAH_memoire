\label{annexe:CEI_schema}

\lstset{style=mystyle}
\begin{lstlisting}[language=XML]
<?xml version="1.0" encoding="UTF-8"?>

<cei xmlns:xsi="http://www.w3.org/2001/XMLSchema-instance" xsi:noNamespaceSchemaLocation="http://www.cei.lmu.de/schema/cei060122.xsd">
		
	<teiHeader>
		<fileDesc>
			<titleStmt>
				<title>Friedrich I. – RI IV,2,2 n. 669</title>
			</titleStmt>
			<editionStmt>
				<p n="volume">RI IV,2,2 - Friedrich I., 2. Lfg. 1158-1168</p>
				<p n="repository">Regesta Imperii Online: <ref type="external" target="http://www.regesta-imperii.de/cei/004-002-002/sources/1159-02-15_1_0_4_2_2_111_669"></ref></p>
			</editionStmt>
			<publicationStmt>
				<p n="authority">Deutsche Kommission für die Bearbeitung der Regesta Imperii e.V. bei der Akademie der Wissenschaften und der Literatur | Mainz</p>
				<publisher>Akademie der Wissenschaften und der Literatur |Mainz – Digitale Akademie</publisher>
				<availability>
					<p>Bereitgestellt unter einer <ref target="https://creativecommons.org/licenses/by/4.0/">Creative Commons Namensnennung (CC BY 4.0)</ref></p>
					<p>Bei Verwendung müssen Sie den Namen des Urhebers und folgenden Link zum Material angeben: <ref type="external" target="http://www.regesta-imperii.de/cei/004-002-002/sources/1159-02-15_1_0_4_2_2_111_669"></ref></p>
				</availability>
				<date></date>
			</publicationStmt>
			<sourceDesc>
				<bibl>
					<idno n="uri">http://www.regesta-imperii.de/id/1159-02-15_1_0_4_2_2_111_669</idno>
					<idno n="department">004</idno>
					<idno n="volume">002</idno>
					<idno n="issue">002</idno>
				</bibl>
			</sourceDesc>
		</fileDesc>
		<encodingDesc>
			<geoDecl xml:id="WGS" datum="WGS84">World Geodetic System</geoDecl>
		</encodingDesc>
		<revisionDesc>
			<p>
				<date type="creation" n="1339417607" value="2012-06-11"></date>
				<date type="lastmod" n="1426241847" value="2015-03-13"></date>
			</p>
		</revisionDesc>
	</teiHeader>

	<charter>
		<chDesc>
			<head>
				<idno>RI IV,2,2 n. 669</idno>
				<issued>
					<issueDate>
						<p>
							<dateRange from="1159-02-15" to="1159-02-15">1159 Februar 15</dateRange>
						</p>
					</issueDate>
					<issuePlace>
						<placeName key="Marengo">Marengo</placeName>
						<location><geo decls="#WGS">44.9167, 8.6667</geo></location>
					</issuePlace>
					<dateOrig>(XV<hi rend="sup">o</hi>  kal. marcii, aput Marengam)</dateOrig>
				</issued>
			</head>
			<relevantPersonal>
					<issuer><persName>Friedrich I.</persName></issuer>
					<recipient><hi rend="spaced">Stadt Asti</hi></recipient>
					<notarius function="chancellor"><hi rend="italic">Reinaldus sacri palacii imperialis canz.</hi></notarius>
					<scribe>verfaßt und geschrieben von RG</scribe>
					<testis function="witnesses">Pfalzgraf Otto (von Wittelsbach), Graf Rudolf von Pfullendorf, Markgraf Obizo Malaspina, Bischof Eberhard von Bamberg und Pfalzgraf Konrad bei Rhein</testis>
			</relevantPersonal>
			<abstract><p>Friedrich nimmt die <span sameAs="recipient"><hi rend="spaced">Stadt Asti</hi> </span>  unter die Herrschaft des Reiches, verleiht den von ihm dort nach freiem Ermessen eingesetzten Rektoren Carioth, Robaldus Gardinus und Petrus Cortessius die namentlich angeführten Regalien (<hi rend="italic">Hec itaque regalia esse dicuntur: moneta, vie publice, aquatica, flumina publica, molandina, furni, forestica, mensure, bancatica, ripatica, portus, argentarie, pedagia, piscationis redditus, sestaria vini et frumenti et eorum, que venduntur ad mensuram, placita, batalia, rubi, restituciones in integrum minorum et alia omnia, que ad regalia iura pertinent</hi>.) in der Stadt, dem Bistum und der Grafschaft mit den genannten Orten vorbehaltlich des königlichen Fodrums gegen einen jährlich an St. Martin (11. November) zu entrichtenden Zins von 150 Mark Silber sowie die Reichsburg Annone mit allen Zugehörungen gegen einen Jahreszins von 50 Mark Silber und sagt zu, die genannten Verleihungen künftig nur den nach seinem Willen aus der Stadt zu erwählenden Amtsträgern zu gewähren. <span n="witnesses">Z.: Pfalzgraf Otto (von Wittelsbach), Graf Rudolf von Pfullendorf, Markgraf Obizo Malaspina, Bischof Eberhard von Bamberg und Pfalzgraf Konrad bei Rhein</span>. - <span n="chancellor"><hi rend="italic">Reinaldus sacri palacii imperialis canz.</hi> </span>; <span n="clerk">verfaßt und geschrieben von RG</span>. <span n="incipit"><hi rend="italic">Universorum Christi et imperii</hi> </span>.</p></abstract>
			<diplomaticAnalysis>
				<div type="incipit"><p><hi rend="italic">Universorum Christi et imperii</hi></p></div>
				<div type="transmission"><p>Kop.: Verstümmelte Abschrift des 14. Jh. im Cod. Astensis f. 4, Stadtarchiv Asti (C); notarielle Abschrift von 1356, Staatsarchiv Turin (D); Abschrift des 17. Jh., Benefizi di qua dai Monti, Miscell. 2 f. 307 ebenda (E). Drucke: Sella, Cod. Astensis 2, 73 n<hi rend="sup">o</hi>  6; MG. DF. I. 259. Reg.: Stumpf 3844.</p></div>
				<div type="commentary"><p>Möglicherweise beriefen sich die Bürger von Asti auf die Zusage weiterer Gunsterweise für den Fall des Italienzuges im DK. III. 60, vgl. die Vorbemerkung zum D. Markgraf Wilhelm von Montferrat (zu seiner früheren Haltung gegenüber Asti vgl. oben Reg. <ref target="http://www.regesta-imperii.de/regesten/4-2-1-friedrich-i/nr/1155-02-01_1_0_4_2_1_274_274.html" type="internal" n="regesten">274</ref>) dürfte sich jedenfalls beim Kaiser zugunsten der Stadt verwendet haben, vgl. Brader, Bonifaz von Montferrat (Hist. Studien 55, 1907) 12 f. - Aus ebendieser Zeit - wohl schon von 1158 - ist ein Treueid der Astenser Bürger für Kaiser Friedrich I. überliefert, Schneider, Neue Dokumente, QFIAB 16 (1914) 24 n<hi rend="sup">o</hi>  5. - Zum D. vgl. auch Opll, Stadt und Reich, 199. - Wohl um eine Verwechslung mit den Ereignissen von 1155 (Reg. <ref target="http://www.regesta-imperii.de/regesten/4-2-1-friedrich-i/nr/1155-02-01_1_0_4_2_1_274_274.html" type="internal" n="regesten">274</ref>) handelt es sich bei dem Bericht des Burchard von Ursberg, ed. Holder-Egger, MG. SS rer. Germ. in us. schol., 32 über eine Einnahme von Asti und Eroberung von Annone um diese Zeit.</p></div>
			</diplomaticAnalysis>
			<div type="resources">
				<list><item><ref target="http://regesta-imperii.digitale-sammlungen.de/seite/ri04_opl1991_0047" type="external" n="bsb">Digitalisat der Buchseite</ref></item><item><ref target="http://www.mgh.de/dmgh/diplomata/resolving/D_F_I_259" type="external" n="unkown">Druck in den MGH Diplomata</ref></item></list>
			</div>
		</chDesc>
	</charter>
</cei>
		

\end{lstlisting}